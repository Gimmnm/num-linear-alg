\documentclass[UTF8]{ctexart}
\usepackage{geometry, CJKutf8}
\geometry{margin=1.5cm, vmargin={0pt,1cm}}
\setlength{\topmargin}{-1cm}
\setlength{\paperheight}{29.7cm}
\setlength{\textheight}{25.3cm}

% useful packages.
\usepackage{amsfonts}
\usepackage{amsmath}
\usepackage{amssymb}
\usepackage{amsthm}
\usepackage{enumerate}
\usepackage{graphicx}
\usepackage{multicol}
\usepackage{fancyhdr}
\usepackage{layout}
\usepackage{listings}
\usepackage{float, caption}
\usepackage{booktabs}  % 提供更好的表格线
\usepackage{xcolor}
\lstset{
  basicstyle=\ttfamily\small,      % 设置字体样式
  numbers=left,                    % 显示行号
  numberstyle=\tiny\color{gray},   % 行号样式
  stepnumber=1,                    % 行号步长
  numbersep=10pt,                  % 行号与代码之间的间距
  backgroundcolor=\color{lightgray!20},  % 设置背景颜色
  frame=single,                    % 添加边框
  breaklines=true,                 % 自动换行
  captionpos=b,                    % 标题位置
}

% some common command
\newcommand{\dif}{\mathrm{d}}
\newcommand{\avg}[1]{\left\langle #1 \right\rangle}
\newcommand{\difFrac}[2]{\frac{\dif #1}{\dif #2}}
\newcommand{\pdfFrac}[2]{\frac{\partial #1}{\partial #2}}
\newcommand{\OFL}{\mathrm{OFL}}
\newcommand{\UFL}{\mathrm{UFL}}
\newcommand{\fl}{\mathrm{fl}}
\newcommand{\op}{\odot}
\newcommand{\Eabs}{E_{\mathrm{abs}}}
\newcommand{\Erel}{E_{\mathrm{rel}}}

\begin{document}

\pagestyle{fancy}
\fancyhead{}
\lhead{高卓杭, 3230104543}
\chead{数值代数第三次作业}
\rhead{Mar.9th, 2025}

\section{程序设计}

本次实验继续在上次的\texttt{Matrix<T>}的类中加入了成员函数

\texttt{std::pair<Matrix<T>, std::vector<T>> LDLTDecomposition() const}

与函数\texttt{Matrix<T> CholeskyDecomposition() const} 分别为改进的平方根分解法和平方根分解法.

第一个函数返回$L$与$D$矩阵,第二个函数直接返回$L$矩阵,在用平方根分解解矩阵的过程中,用到回代法.

\section{测试与实验结果}

对于第一个矩阵,由\texttt{Matrix<T>}类中内置的设置三对角矩阵函数

\texttt{static Matrix<T> TridiagonalMatrix(size\_t n, T subVal, T diagVal, T supVal)}设置

$Hilbert$矩阵由内置的\texttt{static Matrix<T> HilbertMatrix(size\_t n)}设置,

\subsection{三对角矩阵测试与结果}

在三对角矩阵测试中,分别设置了$b$为随机向量以及$b$为使得理论解均为$1$的向量进行测试,测试结果如下:

%-----------------------------------------
% (1) 三对角(100阶) + 随机 b + LU分解
\begin{table}[ht]
  \centering
  \caption{三对角(100阶),随机 $\mathbf{b}$,LU分解:部分解向量示例}
  \label{tab:tri100-randb-LU}
  \begin{tabular}{ccc}
  \toprule
  \textbf{Index} $i$ & $x[i]$  &  解的绝对误差值 \\
  \midrule
  0   &  -0.47789794  & 0.00000000\\
  1   &  6.42029912  & 0.00000000\\
  2   &  10.61363734  &0.00000000\\
  $\vdots$ & $\vdots$ & $\vdots$\\
  97  &  11.55189785  &0.00000000\\
  98  &  10.25040401  &0.00000000\\
  99  &  13.16587475  &0.00000000\\
  \bottomrule
  \end{tabular}
  \end{table}
  
  %-----------------------------------------
  % (2) 三对角(100阶) + 随机 b + 列主元分解
  \begin{table}[ht]
  \centering
  \caption{三对角(100阶),随机 $\mathbf{b}$,列主元分解:部分解向量示例}
  \label{tab:tri100-randb-partialPivot}
  \begin{tabular}{ccc}
  \toprule
  \textbf{Index} $i$ & $x[i]$ & 解的绝对误差值 \\
  \midrule
  0   &  -0.29071664 & 0.00000000 \\
  1   &  12.79312675 & 0.00000000 \\
  2   &  3.75417163 & 0.00000000 \\
  $\vdots$ & $\vdots$ & $\vdots$\\
  97  &  -1.19680859 & 0.00000000 \\
  98  &  8.45954290  & 0.00000000\\
  99  &  5.63816236  & 0.00000000\\
  \bottomrule
  \end{tabular}
  \end{table}
%-----------------------------------------
% (3) 三对角(100阶) + 随机 b + CholeskyDecomposition
\begin{table}[ht]
  \centering
  \caption{三对角(100阶),随机 $\mathbf{b}$,Cholesky 分解:部分解向量示例}
  \label{tab:tri100-randb-cholesky}
  \begin{tabular}{ccc}
  \toprule
  \textbf{Index} $i$ & $x[i]$ & 解的绝对误差值\\
  \midrule
  0   &  5.23611518 & 0.00000000 \\
  1   &  2.00788784  & 0.00000000\\
  2   &  9.69269693  & 0.00000000\\
  $\vdots$ & $\vdots$  & $\vdots$\\
  97  &  1.88352355  & 0.00000000\\
  98  &  10.64939405  & 0.00000000\\
  99  &  0.58921333 & 0.00000000 \\
  \bottomrule
  \end{tabular}
  \end{table}
  
  %-----------------------------------------
  % (4) 三对角(100阶) + 随机 b + 改进的平方根分解 (LDLT)
  \begin{table}[ht]
  \centering
  \caption{三对角(100阶),随机 $\mathbf{b}$,LDL$^T$分解:部分解向量示例}
  \label{tab:tri100-randb-LDLT}
  \begin{tabular}{ccc}
  \toprule
  \textbf{Index} $i$ & $x[i]$ & 解的绝对误差值\\
  \midrule
  0   &  6.24026860  & 0.00000000\\
  1   &  -0.09503936  & 0.00000000\\
  2   &  8.05134473  & 0.00000000\\
  $\vdots$ & $\vdots$ & $\vdots$\\
  97  &  13.52248286  & 0.00000000\\
  98  &  2.27357061  & 0.00000000\\
  99  &  13.88956532  & 0.00000000\\
  \bottomrule
  \end{tabular}
  \end{table}
  
  %-----------------------------------------
  % (5) 三对角(100阶) + b使真解为1 + LU分解
  \begin{table}[ht]
  \centering
  \caption{三对角(100阶),$\mathbf{b}$ 使真解为 1,LU分解:部分解向量示例}
  \label{tab:tri100-onesolution-LU}
  \begin{tabular}{cc}
  \toprule
  \textbf{Index} $i$ & $x[i]$ \\
  \midrule
  0   &  1.00000000  \\
  1   &  1.00000000  \\
  2   &  1.00000000  \\
  $\vdots$ & $\vdots$ \\
  97  &  1.00000000 \\
  98  &  1.00000000  \\
  99  &  1.00000000  \\
  \bottomrule
  \end{tabular}
  \end{table}
  
  %-----------------------------------------
  % (6) 三对角(100阶) + b使真解为1 + 列主元分解
  \begin{table}[ht]
  \centering
  \caption{三对角(100阶),$\mathbf{b}$ 使真解为 1,列主元分解:部分解向量示例}
  \label{tab:tri100-onesolution-partialPivot}
  \begin{tabular}{cc}
  \toprule
  \textbf{Index} $i$ & $x[i]$ \\
  \midrule
  0   &  1.00000000  \\
  1   &  1.00000000 \\
  2   &  1.00000000  \\
  $\vdots$ & $\vdots$ \\
  97  &  1.00000000 \\
  98  &  1.00000000  \\
  99  &  1.00000000  \\
  \bottomrule
  \end{tabular}
  \end{table}
  
  %-----------------------------------------
  % (7) 三对角(100阶) + b使真解为1 + CholeskyDecomposition
  \begin{table}[ht]
  \centering
  \caption{三对角(100阶),$\mathbf{b}$ 使真解为 1,Cholesky 分解:部分解向量示例}
  \label{tab:tri100-onesolution-cholesky}
  \begin{tabular}{cc}
  \toprule
  \textbf{Index} $i$ & $x[i]$ \\
  \midrule
  0   &  1.00000000  \\
  1   &  1.00000000  \\
  2   &  1.00000000 \\
  $\vdots$ & $\vdots$ \\
  97  &  1.00000000  \\
  98  &  1.00000000  \\
  99  &  1.00000000  \\
  \bottomrule
  \end{tabular}
  \end{table}
  
  %-----------------------------------------
  % (8) 三对角(100阶) + b使真解为1 + LDLT
  \begin{table}[ht]
  \centering
  \caption{三对角(100阶),$\mathbf{b}$ 使真解为 1,LDL$^T$ 分解:部分解向量示例}
  \label{tab:tri100-onesolution-LDLT}
  \begin{tabular}{cc}
  \toprule
  \textbf{Index} $i$ & $x[i]$ \\
  \midrule
  0   &  1.00000000 \\
  1   &  1.00000000  \\
  2   &  1.00000000  \\
  $\vdots$ & $\vdots$ \\
  97  &  1.00000000  \\
  98  &  1.00000000  \\
  99  &  1.00000000 \\
  \bottomrule
  \end{tabular}
  \end{table}

可以看到四个算法都得到了非常精确的解,在这个矩阵的求解中,四个算法的精确度被拉平了.

\clearpage

\subsection{Hilbert矩阵测试与结果}

在Hilber矩阵测试中,用平方根分解的过程中,发现待开根号的值非常接近$0$,同时会出现很小的负数的情况,导致完全无法算出结果,故得到了$nan$的结果;同时,在列主元方法和改进的平方根分解中,实际上已经出现了主元小于$tolerance$的情况,列表中是不管主元的大小继续运算得到的结果;四个算法在$Hilber$矩阵的情况下都无法很好地求解,这与$Hilbert$矩阵的条件数过大有关系。下面还给出了用平方根分解求解13阶$Hilbert$矩阵的结果,可以看到在阶数较小的情况下,还是可以一定程度地求解。

%-----------------------------------------
% (9) Hilbert(40阶) + LU
\begin{table}[ht]
  \centering
  \caption{Hilbert(40阶),LU分解:部分解向量示例}
  \label{tab:hilbert40-LU}
  \begin{tabular}{cc}
  \toprule
  \textbf{Index} $i$ & $x[i]$ \\
  \midrule
  0   &  0.99999980  \\
  1   &  1.00003876  \\
  2   &  0.99818251  \\
  $\vdots$ & $\vdots$ \\
  37  & -26.65544669 \\
  38  &  34.07261564 \\
  39  &  -12.43474932 \\
  \bottomrule
  \end{tabular}
  \end{table}
  
  %-----------------------------------------
  % (10) Hilbert(40阶) + 列主元
  \begin{table}[ht]
  \centering
  \caption{Hilbert(40阶),列主元分解:部分解向量示例}
  \label{tab:hilbert40-partialPivot}
  \begin{tabular}{cc}
  \toprule
  \textbf{Index} $i$ & $x[i]$ \\
  \midrule
  0   &  0.99999991  \\
  1   &  1.00001928  \\
  2   &  0.99902559  \\
  $\vdots$ & $\vdots$ \\
  37  & -6.14223246 \\
  38  &  -1.65589767 \\
  39  &  2.81407370 \\
  \bottomrule
  \end{tabular}
  \end{table}
  
  %-----------------------------------------
  % (11) Hilbert(40阶) + Cholesky
  \begin{table}[ht]
  \centering
  \caption{Hilbert(40阶),Cholesky 分解:部分解向量示例}
  \label{tab:hilbert40-cholesky}
  \begin{tabular}{cc}
  \toprule
  \textbf{Index} $i$ & $x[i]$ \\
  \midrule
  0   &  nan \\
  1   &  nan  \\
  2   &  nan  \\
  $\vdots$ & $\vdots$ \\
  37  & nan \\
  38  &  nan\\
  39  &  nan \\
  \bottomrule
  \end{tabular}
  \end{table}
  
  %-----------------------------------------
  % (12) Hilbert(40阶) + LDLT
  \begin{table}[ht]
  \centering
  \caption{Hilbert(40阶),LDL$^T$ 分解:部分解向量示例}
  \label{tab:hilbert40-LDLT}
  \begin{tabular}{cc}
  \toprule
  \textbf{Index} $i$ & $x[i]$ \\
  \midrule
  0   &  1.00000015  \\
  1   &  0.99997161  \\
  2   &  1.00130808  \\
  $\vdots$ & $\vdots$ \\
  37  & -46.42889503 \\
  38  &  73.83362397 \\
  39  &  -22.08116251 \\
  \bottomrule
  \end{tabular}
  \end{table}


  \begin{table}[ht]
    \centering
    \caption{Hilbert(13阶),Cholesky 分解:部分解向量示例}
    \label{tab:hilbert13-cholesky}
    \begin{tabular}{cc}
    \toprule
    \textbf{Index} $i$ & $x[i]$ \\
    \midrule
    0   &  1.00000014 \\
    1   &  0.99997737  \\
    2   &  1.00087297  \\
    $\vdots$ & $\vdots$ \\
    37  & 7.07570132\\
    38  &  -1.30294173\\
    39  &  1.38273717\\
    \bottomrule
    \end{tabular}
    \end{table}
  

\end{document}